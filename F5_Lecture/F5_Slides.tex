% %!TEX TS-program = pdflatex
\documentclass[aspectratio=169,11pt]{beamer}
\usepackage{../eng_beamer}

% Title page information
\title[JRE 300 Finance Lecture \#5]{Finance Lecture \#5: Corporate Finance}
\subtitle{JRE 300: Fundamentals of Accounting and Finance}
\author{Kevin Mott}
\institute{Rotman School of Management\\University of Toronto}
\date{}

\begin{document}

% Title slide
\begin{frame}
\titlepage
\end{frame}

%===============================================================================
% PREVIEW SLIDES

% Slide: Preview of lecture
\begin{frame}{Today's Roadmap}
\textbf{Last week}: We learned how to use CAPM and WACC to evaluate projects.

\vspace{1em}
\textbf{This week}: Capital structure, taxes, central banks, and hedging.

\vspace{1em}
\textbf{Part 1: The Frictionless Benchmark (MM Theorem)}
\begin{itemize}
\item Value comes from assets, not financing: $V = E + D$
\item How leverage affects equity beta and cost of equity
\item WACC stays constant (in a frictionless world)
\end{itemize}

\vspace{1em}
\textbf{Part 2: Taxes and Bankruptcy Costs}
\begin{itemize}
\item Interest tax shield increases firm value
\item Bankruptcy costs destroy value
\item Trade-off theory: balancing tax benefits and distress costs
\item Corporate tax policy and equity market stability
\end{itemize}
\end{frame}

% Slide: Preview parts 3 & 4
\begin{frame}{Today's Roadmap (cont.)}
\textbf{Part 3: Central Banks and the Real Economy}
\begin{itemize}
\item How the policy rate affects all interest rates
\item Transmission mechanism: policy rate $\rightarrow$ WACC $\rightarrow$ investment
\item Why central bank decisions affect firm valuations and hiring
\item The real channel of monetary policy
\end{itemize}

\vspace{1em}
\textbf{Part 4: Introduction to Hedging}
\begin{itemize}
\item Hedging vs.\ speculation
\item Using options to manage risk (airline oil exposure example)
\item Call and put options, payoff diagrams
\item Preview: Options pricing next week
\end{itemize}

\vspace{1em}
\textbf{Big picture}: Connecting firm-level finance decisions to macro policy and risk management.
\end{frame}

%===============================================================================
% PART 1: VALUATION AND VALUE CREATION

\begin{frame}
\begin{center}
{\Huge \textcolor{uoftblue}{Part 1}}\\[1em]
{\Large How to Value a Firm}
\end{center}
\end{frame}

% Slide: What is a firm worth?
\begin{frame}{What is a firm worth?}
\textbf{Fundamental principle}:
\begin{center}
\large
\textcolor{uoftblue}{Financial prices = the present value of future cash flows}
\end{center}

\vspace{1em}
\textbf{For a firm, what are the cash flows?}

\vspace{0.5em}
\begin{itemize}
\item Cash flows to \textcolor{uoftteal}{\textbf{equity holders}}: dividends and capital gains
\item Cash flows to \textcolor{uoftteal}{\textbf{debt holders}}: interest and principal repayments
\end{itemize}

\vspace{1em}
\textbf{How do we measure a firm's value?}
\begin{itemize}
\item We look at the \textbf{market values} of all outstanding claims
\item These market values reflect the PV of expected cash flows to each group of investors
\end{itemize}
\end{frame}

% Slide: Valuing different capital structures
\begin{frame}{Valuing firms with different capital structures}
\textbf{All-equity firm}:
\begin{itemize}
\item Firm value = \textcolor{uoftteal}{\textbf{Stock market capitalization}}
\item Market cap = stock price $\times$ shares outstanding
\item This is the PV of all future cash flows to shareholders
\end{itemize}

\vspace{1em}
\textbf{What if the firm issues debt?}
\begin{itemize}
\item This creates cash flows (interest and principal) that now go to bondholders
\item These are cash flows that would have otherwise gone to the firm's owners
\item So we need to account for them: How much value is this?
\item Answer: \textcolor{uoftteal}{\textbf{Market value of debt}} (what the debt trades for)
\item This reflects the PV of promised payments to bondholders, adjusted for default risk
\end{itemize}

\vspace{1em}
\textbf{If both equity and debt are used}:
\begin{itemize}
\item Firm value = Market value of equity + Market value of debt
\item We sum the PVs of cash flows to each group of investors
\end{itemize}
\end{frame}

% Slide: Firm value formula
\begin{frame}{A firm's total value: $V = E + D$}
\textbf{Our metric for a firm's value}:
\[
\boxed{V = E + D}
\]

\vspace{0.5em}
where:
\begin{itemize}
\item $V$ = firm value
\item $E$ = \textbf{market value} of equity (stock market cap)
\item $D$ = \textbf{market value} of debt (what bonds trade for)
\end{itemize}

\vspace{1em}
\textbf{Why market values?}
\begin{itemize}
\item Market prices reflect investors' assessments of future cash flows
\item Book values (accounting values) are backward-looking 
\item Market values tell us what claims are \emph{actually worth} today
\end{itemize}

\vspace{1em}
\textbf{Key insight}: A firm's value is the sum of the market values of all outstanding claims on its cash flows.
\end{frame}

% Slide: How can a firm change its value?
\begin{frame}{How can a firm change its value?}
\textbf{Question}: What increases $V = E + D$?\\ 
\textbf{Answer}: Doing projects with \textcolor{uoftteal}{\textbf{NPV $>$ 0}}.

\vspace{1em}
\textbf{What do firms do?}
\begin{enumerate}
\item \textbf{Identify productive investment opportunities} (positive NPV projects)
\begin{itemize}
    \item New factories, R\&D, product launches, acquisitions
    \item These projects generate cash flows worth more than they cost
\end{itemize}

\vspace{0.5em}
\item \textbf{Raise financing to afford these investments}
\begin{itemize}
    \item Issue debt
    \item Issue equity or use retained earnings
\end{itemize}
\end{enumerate}

\vspace{0.5em}
\textbf{The big picture}:
\begin{itemize}
\item Financial markets link excess capital (from savers/investors) to firms with productive uses for it
\item This creates \textcolor{uoftteal}{\textbf{financial value}} (returns) and \textcolor{uoftteal}{\textbf{economic growth}} (real output)
\end{itemize}
\end{frame}

% Slide: The puzzle
\begin{frame}{A puzzle: Does financing affect NPV?}
\textbf{We learned last week}:
\begin{itemize}
\item Cost of debt $<$ Cost of equity ($r_D < r_E$)
\item Debt is cheaper because it's less risky (fixed payments, paid first in bankruptcy)
\end{itemize}

\vspace{1em}
\textbf{Natural questions}:
\begin{enumerate}
    \item \textcolor{uoftblue}{If debt financing is cheaper, would using more debt raise project NPVs?}
    \item \textcolor{uoftblue}{Why is there even a stock market if debt is cheaper?}
\end{enumerate}

\vspace{1em}
\textbf{This seems like it should matter!}
\begin{itemize}
\item If we can finance projects more cheaply, doesn't that increase their value?
\item Should all firms just use 100\% debt financing?
\end{itemize}
\end{frame}

% Slide: Preview the answer
\begin{frame}{Preview: Financing shouldn't matter}
\textbf{The answer (spoiler alert)}:
\begin{center}
\large
\textcolor{uoftblue}{No, financing choice doesn't affect NPV or firm value}
\end{center}

\vspace{0.5em}
\textbf{Important}: For now, assume \textcolor{red}{\textbf{no taxes}} and \textcolor{red}{\textbf{no bankruptcy costs}} (frictionless world).

\vspace{1em}
\textbf{Why doesn't financing matter?}
\begin{itemize}
\item Yes, debt is cheaper than equity ($r_D < r_E$)
\item But when you add more debt, equity becomes riskier
\item Equity holders demand a higher return to compensate for the extra risk
\item The cost of equity \textcolor{uoftteal}{\textbf{equilibrates}} to the debt level
\item These forces exactly offset each other
\end{itemize}

\vspace{1em}
\textbf{Result}: WACC stays constant regardless of capital structure

\vspace{0.5em}
Let's build the intuition for why this happens.
\end{frame}

% Slide: Why WACC stays constant
% \begin{frame}{Why WACC stays constant: The offsetting forces}
% \textbf{Recall from last week}:
% \[
% \text{WACC} = \frac{E}{E+D} \times r_E + \frac{D}{E+D} \times r_D \times (1-\tau_c)
% \]

% \vspace{1em}
% \textbf{What happens when we increase debt?}

% \vspace{0.5em}
% \begin{itemize}
% \item \textcolor{red}{\textbf{Force 1}}: We're using more debt, which is cheaper ($r_D < r_E$)
% \begin{itemize}
%     \item This \emph{lowers} WACC
% \end{itemize}

% \vspace{0.3em}
% \item \textcolor{uoftteal}{\textbf{Force 2}}: But equity becomes riskier (more leverage)
% \begin{itemize}
%     \item Equity holders now have a riskier residual claim
%     \item They demand a higher return: $r_E$ increases
%     \item This \emph{raises} WACC
% \end{itemize}
% \end{itemize}

% \vspace{1em}
% \textbf{Key insight}:
% \begin{center}
% These two forces \textcolor{uoftblue}{\textbf{exactly offset}} each other (ignoring taxes and bankruptcy costs)
% \end{center}

% \vspace{0.5em}
% This is why WACC stays constant, and why capital structure doesn't create value by itself.
% \end{frame}

%===============================================================================
% BRIDGE: PROJECT BETA VS EQUITY BETA

% Slide: Understanding project risk vs equity risk
\begin{frame}{Understanding the hurdle rate: Project beta vs. Equity beta}

When we evaluate a project, the discount rate depends on the \textcolor{uoftteal}{\textbf{project's risk}}, not the firm's equity risk. 

\vspace{1em}
The project's systematic risk determines its required return (the hurdle rate). We often use \textcolor{uoftteal}{\textbf{comparables}} (industry averages) to estimate project beta. 

\vspace{1em}
But there's a complication: 

\vspace{1em}
\textcolor{uoftblue}{The beta of the \emph{project} is different from the beta of the \emph{equity}} 

\vspace{1em}
Let's see why this matters, starting with the simplest case.

\end{frame}

% Slide: All-equity firm case
\begin{frame}{Starting point: The all-equity firm}
Consider a firm with no debt (100\% equity financed).

\vspace{1em}
All cash flows from the project go to equity holders, so they bear all the project's systematic risk.

\vspace{1em}
Therefore: \textcolor{uoftteal}{\textbf{$\beta_E = \beta_{\text{Project}}$}}

\vspace{1.5em}
\textbf{Example}: Suppose a firm's projects have systematic risk $\beta_{\text{Project}} = 1.2$

\vspace{0.5em}
With no debt, equity holders bear all this risk, so $\beta_E = 1.2$.

\vspace{1.5em}
Simple so far. When there's no debt, equity beta equals project beta.

\vspace{1em}
Now let's add debt\ldots
\end{frame}

% Slide: Adding debt
\begin{frame}{What happens when we add debt?}
Suppose the firm issues debt (assume debt is risk-free: $\beta_D = 0$).

\vspace{1em}
Does the project's systematic risk change? \textcolor{red}{\textbf{No!}} The project's risk is determined by its cash flows, not how it's financed.

\vspace{1em}
But here's the problem:
\begin{itemize}
\item The project still has systematic risk $\beta_{\text{Project}} = 1.2$
\item Debt holders now bear some of the firm's value, but \emph{zero} systematic risk ($\beta_D = 0$)
\item Regardless of project outcomes, debtholders must be repaid---no exposure to systematic risk
\item Equityholders now have more downside when things go wrong, more upside when things go well
\item Their exposure to systematic risk \emph{rises}
\end{itemize}

\vspace{1em}
For the project beta to stay constant, $\beta_E$ must \textbf{rise}.
\end{frame}

% Slide: The beta conservation equation
\begin{frame}{Beta conservation: Why $\beta_E$ must rise}
The firm's overall systematic risk is the weighted average of debt and equity risk:
\[
\beta_{\text{Project}} = \frac{E}{E+D} \times \beta_E + \frac{D}{E+D} \times \beta_D
\]

\vspace{0.5em}
Solving for $\beta_E$:
\[\beta_E = \beta_{\text{Project}} + \frac{D}{E} \times (\beta_{\text{Project}} - \beta_D)\]

\vspace{0.5em}
With risk-free debt ($\beta_D = 0$): $\beta_E = \beta_{\text{Project}} \times (1 + D/E)$

\vspace{0.5em}
As leverage ratio $D/E$ increases, $\beta_E$ must increase proportionally to keep $\beta_{\text{Project}}$ constant. Equity holders now bear \emph{amplified} systematic risk.
\end{frame}

% Slide: Numerical example
\begin{frame}{Example: How leverage amplifies equity beta}
Project beta: $\beta_{\text{Project}} = 1.2$, debt is risk-free: $\beta_D = 0$

\vspace{.5em}
\textbf{All-equity firm} ($D/E = 0$):
\[
\beta_E = 1.2 \times (1 + 0) = 1.2
\]

\vspace{.5em}
\textbf{Add debt} to achieve $D/E = 0.5$:
\[
\beta_E = 1.2 \times (1 + 0.5) = 1.8
\]

\vspace{.5em}
\textbf{More debt}, $D/E = 1$:
\[
\beta_E = 1.2 \times (1 + 1) = 2.4
\]

\vspace{.5em}
Even though the project's risk hasn't changed ($\beta_{\text{Project}} = 1.2$ throughout), equity beta doubles from 1.2 to 2.4 as leverage increases.
\end{frame}

% Slide: Connecting to cost of equity
\begin{frame}{Cost of equity rises with leverage}
We just showed: $\beta_E = \beta_{\text{Project}} + \frac{D}{E}(\beta_{\text{Project}} - \beta_D)$

\vspace{1em}
Using CAPM for equity and debt:
\begin{align*}
r_E &= r_f + \beta_E \times (\mathbb{E}[R_m] - r_f) \\[0.5em]
&= r_f + \left[\beta_{\text{Project}} + \frac{D}{E}(\beta_{\text{Project}} - \beta_D)\right] \times (\mathbb{E}[R_m] - r_f) \\[0.5em]
&= r_f + \beta_{\text{Project}} \times (\mathbb{E}[R_m] - r_f) + \frac{D}{E}(\beta_{\text{Project}} - \beta_D) \times (\mathbb{E}[R_m] - r_f) \\[0.5em]
&= r_0 + \frac{D}{E}\left[\beta_{\text{Project}} \times (\mathbb{E}[R_m] - r_f) - \beta_D \times (\mathbb{E}[R_m] - r_f)\right] \\[0.5em]
&= r_0 + \frac{D}{E}\left[(r_0 - r_f) - (r_D - r_f)\right] = \boxed{r_0 + \frac{D}{E}(r_0 - r_D)}
\end{align*}
\end{frame}

% Slide: Interpreting the formula
\begin{frame}{Interpreting the cost of equity formula}
\[
\underbrace{r_E}_{\text{Levered COE}} = \underbrace{r_0}_{\text{Unlevered COE}} + \underbrace{(r_0 - r_D) \times \frac{D}{E}}_{\text{Leverage premium}}
\]

\textbf{Unlevered cost of equity} ($r_0$): The return equity holders require when the firm has no debt. This compensates them for the project's systematic risk.

\vspace{1em}
\textbf{Leverage premium} $(r_0 - r_D) \times D/E$: The \emph{additional} return equity holders demand to compensate for increased risk exposure from financial leverage.

\vspace{1em}
As debt increases, equity holders bear amplified systematic risk (the residual claim gets riskier). The leverage premium grows proportionally with $D/E$ to compensate for this increased exposure.

\vspace{0.5em}
Note: With risk-free debt ($r_D = r_f$), this becomes $r_E = r_0 + (r_0 - r_f) \times D/E$.
\end{frame}

% Slide: What does this mean for WACC?
\begin{frame}{What does this mean for WACC?}
WACC is the \textcolor{uoftteal}{\textbf{weighted average opportunity cost}} of the pool of investors in a project.

\vspace{1em}
\begin{center}
\large
\textcolor{uoftblue}{WACC equilibrates to the risk level of the \emph{project}, not the equity}
\end{center}

\vspace{1em}
The project has a fixed systematic risk ($\beta_{\text{Project}}$) that determines its required return. As we change $D/E$, we're just \textbf{reallocating risk} among investors.

\vspace{1em}
\textcolor{uoftteal}{\textbf{Cost of equity}} and \textcolor{uoftteal}{\textbf{cost of debt}} are compensation to investors according to their \emph{exposure}.

\vspace{1.5em}
Higher debt $\rightarrow$ equity bears more risk $\rightarrow$ $r_E$ rises. But debt is cheaper, and we're using more of it. These forces offset.

\vspace{1em}
\textcolor{uoftblue}{\textbf{WACC stays constant at the project's risk level.}}

\vspace{1em}
Capital structure doesn't change project value---it just changes how risk and returns are distributed among investors.
\end{frame}

% Slide: Plugging into WACC
\begin{frame}{What happens when we plug this into WACC?}
Start with WACC and substitute our formula for $r_E$:

\begin{align*}
\text{WACC} &= \frac{E}{E+D} \times r_E + \frac{D}{E+D} \times r_D \\[0.5em]
&= \frac{E}{E+D} \times \left[r_0 + (r_0 - r_D) \times \frac{D}{E}\right] + \frac{D}{E+D} \times r_D \\[0.5em]
&= \frac{E}{E+D} \times r_0 + \frac{D}{E+D} \times (r_0 - r_D) + \frac{D}{E+D} \times r_D \\[0.5em]
&= \frac{E}{E+D} \times r_0 + \frac{D}{E+D} \times r_0 \\[0.5em]
&= \boxed{r_0}
\end{align*}

\vspace{0.5em}
WACC equals the unlevered cost of equity, regardless of leverage! The rising cost of equity exactly offsets the use of cheaper debt.
\end{frame}

% Slide: Visual representation of cost of capital vs leverage
\begin{frame}{Capital Structure and Cost of Capital (No Tax)}
\vspace{0.5em}
\begin{center}
\begin{tikzpicture}[scale=.85]
% Axes
\draw[->] (0,0) -- (8,0) node[right] {Debt-to-Equity Ratio ($D/E$)};
\draw[->] (0,0) -- (0,6) node[above] {Cost of Capital};

% r_D line (constant at 3% = 0.6 units)
\draw[very thick, uoftblue] (0,0.6) -- (7.5,0.6);
\node[right, uoftblue] at (7.5,0.6) {$r_D$};

% r_0 / WACC line (constant at 11.4% = 2.28 units)
\node[left, uoftteal] at (0,2.28) {$r_0$};
\draw[very thick, uoftteal] (0,2.28) -- (7.5,2.28);
\node[right, uoftteal] at (7.5,2.28) {WACC};

% r_E line (starts at r_0, increases with slope (r_0 - r_D))
% At D/E=0: r_E = 11.4% = 2.28
% At D/E=1: r_E = 19.8% = 3.96
% At D/E=2: r_E = 28.2% = 5.64
\draw[very thick, red] (0,2.28) -- (7.5,5.82);
\node[right, red] at (7.5,5.82) {$r_E = r_0 + \frac{D}{E}(r_0-r_D)$};

\end{tikzpicture}
\end{center}

Cost of equity rises with leverage, but WACC stays constant
\end{frame}

% Slide: Numerical example of WACC invariance
\begin{frame}{Example: WACC stays constant as leverage changes}
\textbf{Setup}: $r_f = 3\%$, market risk premium $= 7\%$, $\beta_{\text{Project}} = 1.2$, risk-free debt ($\beta_D = 0$) at $r_D = 3\%$

\vspace{1em}
\textbf{All-equity firm} ($D/E = 0$):
\begin{itemize}
\item $\beta_E = 1.2 \times (1 + 0) = 1.2$
\item $r_E = 3\% + 1.2 \times 7\% = 11.4\%$
\item $\text{WACC} = 100\% \times 11.4\% + 0\% \times 3\% = \textcolor{uoftteal}{\mathbf{11.4\%}}$
\end{itemize}

\vspace{1em}
\textbf{Now add debt} to $D/E = 1$ (50\% debt, 50\% equity):
\begin{itemize}
\item $\beta_E = 1.2 \times (1 + 1) = 2.4$
\item $r_E = 3\% + 2.4 \times 7\% = 19.8\%$
\item $\text{WACC} = 50\% \times 19.8\% + 50\% \times 3\% = 9.9\% + 1.5\% = \textcolor{uoftteal}{\mathbf{11.4\%}}$
\end{itemize}

\vspace{1em}
Cost of equity nearly doubled (from 11.4\% to 19.8\%), but WACC stayed constant because we're using cheaper debt.
\end{frame}

% Slide: Part 1 Summary
\begin{frame}{Summary: The Modigliani-Miller Theorem}
\textbf{What we've shown}:
\begin{itemize}
\item Capital structure is \textcolor{uoftblue}{\textbf{irrelevant}} to firm value in a frictionless world
\item Leverage increases equity beta: $\beta_E = \beta_{\text{Project}} + \frac{D}{E}(\beta_{\text{Project}} - \beta_D)$
\item Cost of equity rises with leverage: $r_E = r_0 + (r_0 - r_D) \times D/E$
\item But WACC stays constant: $\text{WACC} = r_0$ regardless of $D/E$
\item Changing capital structure just \textbf{reallocates risk} among investors
\end{itemize}

\vspace{1em}
This is the \textcolor{uoftteal}{\textbf{Modigliani-Miller Theorem}} (1958)---one of the most important results in finance.

\vspace{1em}
\textbf{Key assumptions we made}:
\begin{itemize}
\item \textcolor{red}{\textbf{No taxes}} (no interest tax shield)
\item \textcolor{red}{\textbf{No bankruptcy costs}}
\end{itemize}

\vspace{0.5em}
Part 2 will relax these assumptions and see when capital structure \emph{does} matter.
\end{frame}

%===============================================================================
% PART 2: REAL-WORLD FRICTIONS

\begin{frame}
\begin{center}
{\Huge \textcolor{uoftblue}{Part 2}}\\[1em]
{\Large Real-World Frictions:\\Taxes and Bankruptcy Costs}
\end{center}
\end{frame}

% Slide: What we missed
\begin{frame}{What our analysis missed}
In Part 1, we showed that capital structure doesn't matter in a frictionless world.

\vspace{1em}
But we made an important simplifying assumption: \textcolor{red}{\textbf{no taxes}}.

\vspace{1em}
In reality, there's a crucial asymmetry in the tax code:

\vspace{0.5em}
\begin{center}
\large
\textcolor{uoftblue}{Interest payments on debt are \textbf{tax-deductible}}
\end{center}

\vspace{1em}
Dividend payments and retained earnings (equity financing) are \emph{not} tax-deductible.

\vspace{1em}
This creates a significant advantage for debt financing. Let's see how.
\end{frame}

% Slide: Tax shield example
\begin{frame}{Example: The tax shield in action}
\textbf{Consider two identical firms}:
\begin{itemize}
\item Both have EBIT = \$100 per year
\item Corporate tax rate $\tau_c = 25\%$
\item \textbf{Firm A}: No debt
\item \textbf{Firm B}: \$500 of debt at 5\% interest
\end{itemize}

\vspace{1em}
\begin{center}
\begin{tabular}{lcc}
\toprule
& \textbf{Firm A (No Debt)} & \textbf{Firm B (Debt)} \\
\midrule
EBIT & \$100 & \$100 \\
Interest & \$0 & \$25 \\
Taxable income & \$100 & \$75 \\
Taxes (25\%) & \$25 & \$18.75 \\
\midrule
\textbf{Cash to investors} & \textbf{\$75} & \textbf{\$81.25} \\
\bottomrule
\end{tabular}
\end{center}

\vspace{1em}
Firm B pays \$6.25 less in taxes. This is the \textcolor{uoftteal}{\textbf{tax shield}} = Interest $\times$ $\tau_c$ = \$25 $\times$ 0.25 = \$6.25.
\end{frame}

% Slide: WACC with taxes
\begin{frame}{Redoing the analysis: WACC with taxes}
In Part 1, we had:
\[
\text{WACC} = \frac{E}{E+D} \times r_E + \frac{D}{E+D} \times r_D
\]

\vspace{1em}
But this ignored the tax deductibility of interest. The \emph{after-tax} cost of debt is lower:
\[
\text{After-tax cost of debt} = r_D \times (1 - \tau_c)
\]

\vspace{1em}
So the correct WACC formula is:
\[
\boxed{\text{WACC} = \frac{E}{E+D} \times r_E + \frac{D}{E+D} \times r_D \times (1 - \tau_c)}
\]

\vspace{1em}
The $(1 - \tau_c)$ term captures the tax shield. Since $\tau_c > 0$, this lowers WACC.
\end{frame}

% Slide: Equity beta with taxes
\begin{frame}{Equity beta decreases (compared to no-tax case)}
The equity beta relationship changes:

\vspace{1em}
\textbf{Without taxes} (from Part 1):
\[
\beta_E = \beta_{\text{Project}} + \frac{D}{E}(\beta_{\text{Project}} - \beta_D)
\]

\vspace{1em}
\textbf{With taxes}, the effective leverage is lower because of the tax shield:
\[
\beta_E = \beta_{\text{Project}} + \frac{D}{E}(\beta_{\text{Project}} - \beta_D) \times (1 - \tau_c)
\]

\vspace{1em}
The $(1 - \tau_c)$ term captures how the tax shield reduces equity holders' systematic risk exposure.
\end{frame}

% Slide: Cost of equity with taxes
\begin{frame}{Cost of equity with taxes (MM Proposition II)}
Starting from CAPM with the adjusted beta:
\begin{align*}
r_E &= r_f + \beta_E \times (\mathbb{E}[R_m] - r_f) \\[0.5em]
&= r_f + \left[\beta_{\text{Project}} + \frac{D}{E}(\beta_{\text{Project}} - \beta_D)(1 - \tau_c)\right] \times (\mathbb{E}[R_m] - r_f) \\[0.5em]
&= r_0 + \frac{D}{E}(r_0 - r_D)(1 - \tau_c)
\end{align*}
\textbf{Result}:
\[
\boxed{r_E = r_0 + (r_0 - r_D) \times \frac{D}{E} \times (1 - \tau_c)}
\]
The leverage premium is \textbf{smaller} with taxes because the tax shield cushions equity holders from systematic risk.
\end{frame}

% Slide: Deriving WACC with taxes
\begin{frame}{What is WACC with the new cost of equity?}
We have: $r_E = r_0 + (r_0 - r_D) \times \frac{D}{E} \times (1 - \tau_c)$

\vspace{1em}
Plugging into WACC:

\begin{align*}
\text{WACC} &= \frac{E}{E+D} \times r_E + \frac{D}{E+D} \times r_D \times (1 - \tau_c) \\[0.5em]
&= \frac{E}{E+D} \times \left[r_0 + (r_0 - r_D) \times \frac{D}{E} \times (1 - \tau_c)\right] + \frac{D}{E+D} \times r_D \times (1 - \tau_c) \\[0.5em]
&= \frac{E}{E+D} \times r_0 + \frac{D}{E+D} \times (r_0 - r_D) \times (1 - \tau_c) + \frac{D}{E+D} \times r_D \times (1 - \tau_c) \\[0.5em]
&= \frac{E}{E+D} \times r_0 + \frac{D}{E+D} \times (1 - \tau_c) \times [(r_0 - r_D) + r_D] \\[0.5em]
&= \frac{E}{E+D} \times r_0 + \frac{D}{E+D} \times (1 - \tau_c) \times r_0 \\[0.5em]
&= r_0 \times \left[\frac{E}{E+D} + \frac{D}{E+D} \times (1 - \tau_c)\right]
\end{align*}
\end{frame}

% Slide: WACC simplifies
\begin{frame}{WACC with taxes}
\begin{align*}
\text{WACC} &= r_0 \times \left[\frac{E}{E+D} + \frac{D}{E+D} \times (1 - \tau_c)\right] \\[0.5em]
&= r_0 \times \left[\frac{E}{E+D} + \frac{D}{E+D} - \frac{D}{E+D} \times \tau_c\right] \\[0.5em]
&= \boxed{r_0 \times \left[1 - \frac{D}{E+D} \times \tau_c\right]}
\end{align*}

\vspace{1em}
As leverage increases ($D \uparrow$), WACC decreases.

\vspace{0.5em}
\textbf{Why lower WACC increases firm value}:
\begin{itemize}
\item Discount future cash flows by less $\rightarrow$ higher present value
\item More projects have positive NPV (lower hurdle rate)
\end{itemize}
\end{frame}

\begin{frame}{Capital Structure and Cost of Capital (Taxes)}
\vspace{0.5em}
\begin{center}
\begin{tikzpicture}[scale=.85]
% Axes
\draw[->] (0,0) -- (8,0) node[right] {Debt-to-Equity Ratio ($D/E$)};
\draw[->] (0,0) -- (0,6) node[above] {Cost of Capital};

% r_D line (constant at 3% = 0.6 units)
\draw[very thick, uoftblue] (0,0.6) -- (7.5,0.6);
\node[right, uoftblue] at (7.5,0.6) {$r_D$};

% r_0 / WACC line (declining toward r_D as leverage increases)
\node[left, uoftteal] at (0,2.28) {$r_0$};
% WACC curve: starts at r_0 = 2.28, declines toward r_D*(1-tau_c)
% Using a smooth curve that asymptotically approaches r_D region
\draw[very thick, uoftteal, domain=0:7.5, smooth, samples=100]
    plot (\x, {2.28 - 1.4*(1 - exp(-0.4*\x))});
\node[right, uoftteal] at (7.5,1.0) {WACC $= r_0\times\left[1-\frac{D}{E+D}\tau_c\right]$};

% r_E line (starts at r_0, increases with slope (r_0 - r_D))
% At D/E=0: r_E = 11.4% = 2.28
% At D/E=1: r_E = 19.8% = 3.96
% At D/E=2: r_E = 28.2% = 5.64
\draw[very thick, red] (0,2.28) -- (7.5,5.2);
\node[right, red] at (7.5,5.2) {$r_E = r_0 + \frac{D}{E}(r_0-r_D)(1-\tau_c)$};

\end{tikzpicture}
\end{center}

Cost of equity rises with leverage, WACC falls (tax shield savings)
\end{frame}

% Slide: So why not 100% debt?
\begin{frame}{So why not finance with 100\% debt?}
\textbf{We just showed}: More debt lowers WACC.

\vspace{1em}
\textbf{This means}:
\begin{itemize}
\item Discount future cash flows at a lower rate $\rightarrow$ higher firm value
\item Lower hurdle rate $\rightarrow$ more projects have positive NPV
\item Greater investment opportunities for value-creating projects
\end{itemize}

\vspace{1em}
\textbf{So why not use 100\% debt?}

\vspace{1em}
\begin{center}
\textcolor{red}{\Large \textbf{Bankruptcy costs}}
\end{center}
\end{frame}

% Slide: Bankruptcy costs destroy value
\begin{frame}{Bankruptcy costs destroy value}
Too much debt increases the probability of \textbf{financial distress} and \textbf{bankruptcy}.

\vspace{1em}
These costs can destroy significant value, offsetting the tax benefits of debt.

\vspace{1em}
The optimal capital structure balances the \textcolor{uoftteal}{\textbf{tax benefits}} of debt against the \textcolor{red}{\textbf{bankruptcy costs}} of too much debt.
\end{frame}

% Slide: What is bankruptcy?
\begin{frame}{What is bankruptcy?}
\textbf{Common misconception}: Bankruptcy = business shuts down.\\
\textbf{Reality}: Bankruptcy is a \textcolor{uoftteal}{\textbf{legal process}} for handling financial distress (cash flows insufficient to pay interest obligations).

\vspace{1em}
\textbf{Two main types}:
\begin{enumerate}
\item \textbf{Reorganization} (Chapter 11 in the US):
\begin{itemize}
    \item Firm continues operating
    \item Renegotiates debt with creditors under court supervision
    \item Emerges with new capital structure (often converting debt to equity)
    \item Examples: General Motors, American Airlines
\end{itemize}
\item \textbf{Liquidation} (Chapter 7 in the US):
\begin{itemize}
    \item Firm ceases operations
    \item Assets sold off to pay creditors
    \item Firm dissolves
\end{itemize}
\end{enumerate}

\vspace{0.5em}
Most large firms file for reorganization, not liquidation.
\end{frame}

% Slide: The bankruptcy mechanism
\begin{frame}{How does bankruptcy risk affect firms?}
\textbf{The mechanism}: When market participants expect higher probability of bankruptcy:

\vspace{1em}
\textbf{Lenders demand compensation for default risk}:
\begin{itemize}
\item If you might not repay, lenders won't lend at the risk-free rate
\item They charge a \textcolor{red}{\textbf{higher cost of debt}} ($r_D \uparrow$)
\item Equivalently: they lend at a discount (pay less than face value for bonds)
\end{itemize}

\vspace{1em}
\textbf{This creates a feedback loop}:
\begin{enumerate}
\item More debt $\rightarrow$ higher bankruptcy probability
\item Higher bankruptcy risk $\rightarrow$ lenders charge higher $r_D$
\item Higher $r_D$ $\rightarrow$ harder to service debt $\rightarrow$ even higher bankruptcy risk
\end{enumerate}

\vspace{1em}
\textcolor{uoftblue}{\textbf{Key insight}}: The \emph{expectation} of bankruptcy raises borrowing costs, which itself increases financial distress---even before actual bankruptcy occurs.
\end{frame}

% Slide: Types of bankruptcy costs
\begin{frame}{Two types of bankruptcy costs}
\textbf{1. Direct bankruptcy costs}:
\begin{itemize}
\item Legal fees, accounting fees, court costs
\item Administrative expenses of reorganization
\item Typically 3-7\% of firm value for large firms
\end{itemize}

\vspace{1em}
\textbf{2. Indirect costs (financial distress)}:
\begin{itemize}
\item \textbf{Lost customers}: worried about warranties, future service
\item \textbf{Lost suppliers}: demand cash up front, refuse credit
\item \textbf{Lost employees}: talented workers leave for more stable firms
\item \textbf{Underinvestment}: can't raise capital for positive NPV projects
\item \textbf{Asset fire sales}: forced to sell assets below fair value
\end{itemize}

\vspace{1em}
Indirect costs are often \textcolor{red}{\textbf{much larger}} than direct costs and begin \emph{before} actual bankruptcy.
\end{frame}

% Slide: Financial distress example
\begin{frame}{Example: Financial distress costs}
\textbf{Consider an airline with high debt levels}:

\vspace{1em}
When the airline appears financially troubled:
\begin{itemize}
\item \textbf{Customers} stop buying tickets (fear airline will shut down)
\item \textbf{Suppliers} demand cash payment for fuel and parts
\item \textbf{Pilots and mechanics} leave for more stable airlines
\item \textbf{Lessors} refuse to lease new aircraft
\end{itemize}

\vspace{1em}
\textbf{Result}:
\begin{itemize}
\item Revenue collapses, costs spike
\item The firm may fail even if it would have been viable with less debt
\item Self-fulfilling prophecy: high debt creates fragility
\end{itemize}
\end{frame}

% Slide: The trade-off theory
\begin{frame}{The trade-off theory of capital structure}
\textbf{The optimization problem} (loose formulation):

\vspace{1em}
\[
\max_{D/E} \quad \sum_{t=1}^{\infty} \frac{CF_t}{\text{WACC}^t} - \text{PV(Bankruptcy Costs)}
\]

\vspace{0.5em}
subject to:
\[
\text{WACC} = r_0 \times \left[1 - \frac{D}{E+D} \times \tau_c\right]
\]

\vspace{1em}
\textbf{Trade-off}:
\begin{itemize}
\item Higher $D/E$ $\rightarrow$ lower WACC $\rightarrow$ higher PV of cash flows and tax shield
\item But higher $D/E$ $\rightarrow$ higher PV(Bankruptcy Costs) $\rightarrow$ destroys value
\end{itemize}
\end{frame}

\begin{frame}{Capital Structure and Cost of Capital (With Bankruptcy Costs)}
\begin{center}
\begin{tikzpicture}[scale=.8]
% Axes
\draw[->] (0,0) -- (8,0) node[right] {Debt-to-Equity Ratio ($D/E$)};
\draw[->] (0,0) -- (0,6) node[above] {Cost of Capital};

% Threshold marker (vertical dashed line)
\draw[dashed, gray] (5,0) -- (5,5.5);
\node[below, gray, font=\small] at (5,-0.3) {Distress region $\to$};

% r_D line (flat then increasing after threshold)
\draw[very thick, uoftblue] (0,0.6) -- (5,0.6);
\draw[very thick, uoftblue, domain=5:7.5, smooth, samples=50]
    plot (\x, {0.6 + 0.4*(\x-5)^1.5});
\node[right, uoftblue] at (7.5,1.8) {$r_D$};

% r_0 starting point
\node[left, uoftteal] at (0,2.28) {$r_0$};

% WACC curve: starts at r_0, declines exponentially (tax shield), then increases after distress threshold
% Part 1: Before distress (0 to 5) - decreases like the no-bankruptcy case
\draw[very thick, uoftteal, domain=0:5, smooth, samples=100]
    plot (\x, {2.28 - 1.4*(1 - exp(-0.4*\x))});
% Part 2: After distress (5 to 7.5) - increases due to bankruptcy costs
\draw[very thick, uoftteal, domain=5:7.5, smooth, samples=50]
    plot (\x, {2.28 - 1.4*(1 - exp(-0.4*\x)) + 0.4*(\x-5)^1.8});
\node[above, uoftteal] at (5,1.05) {WACC};

% r_E line (starts at r_0, increases with leverage, steeper after threshold)
\draw[very thick, red, domain=0:5, smooth, samples=50]
    plot (\x, {2.28 + 0.58*\x});
\draw[very thick, red, domain=5:7.5, smooth, samples=50]
    plot (\x, {5.18 + 0.75*(\x-5)});
\node[right, red] at (7.5,6.0) {$r_E$};

\end{tikzpicture}
\end{center}

Initially: tax shield lowers WACC. Beyond threshold: bankruptcy costs dominate
\end{frame}

% Slide: The challenge of measurement
\begin{frame}{How hard is this to measure in practice?}
\textbf{Very hard.} This is an active area of study in academic corporate finance.

\vspace{1em}
\textbf{The challenges}:
\begin{enumerate}
\item \textbf{Corporate Taxation}: 
\begin{itemize}
    \item Depends on tax rates: corporate tax code is extremely complex
    \item Future tax rates and debt policy are uncertain (elections?)
\end{itemize}

\vspace{0.5em}
\item \textbf{PV(Bankruptcy Costs)}: 
\begin{itemize}
    \item What is the probability of bankruptcy? (model-dependent)
    \item How large are indirect costs? (hard to observe, vary by industry)
    \item How do bankruptcy costs vary with systematic risk and macro conditions?
\end{itemize}

\item \textbf{Other Frictions:}
\begin{itemize}
    \item Capital adjustment costs: it's expensive to move capital around
    \item Raising additional equity: costly and negative signal
\end{itemize}
\end{enumerate}
\end{frame}

% Slide: Mini summary
\begin{frame}{What should managers do?}
\textbf{Bottom line for capital structure choice}:

\vspace{1em}
Managers should choose the debt-to-equity ratio that balances:

\vspace{0.5em}
\begin{enumerate}
\item \textcolor{uoftteal}{\textbf{Reducing WACC}} through the interest tax shield
\item \textcolor{red}{\textbf{Bankruptcy costs}} from financial distress
\item \textcolor{uoftblue}{\textbf{Levered equity risk}} borne by shareholders
\end{enumerate}

\vspace{1em}
This optimization depends heavily on:
\begin{itemize}
\item The firm's systematic risk ($\beta_A$)
\item Industry characteristics (cash flow stability, asset tangibility)
\item The macroeconomic environment
\item \textbf{Corporate tax policy}
\end{itemize}

\vspace{1em}
\textbf{Speaking of taxes}\ldots let's dig deeper into corporate tax policy.

\vspace{0.5em}
(Sounds boring, is actually \textbf{super interesting}.)
\end{frame}






% Slide: Interpreting the tax effect on equity
\begin{frame}{What do taxes do to equity holders?}
Recall: $\beta_E = \beta_{\text{Project}} + \frac{D}{E}(\beta_{\text{Project}} - \beta_D) \times (1 - \tau_c)$

\vspace{1em}
\textbf{Higher} corporate tax rate $\tau_c$ means:

\vspace{0.5em}
\textbf{1. Lower systematic risk exposure}:
\begin{itemize}
\item The $(1 - \tau_c)$ term dampens how much leverage amplifies equity beta
\item Tax shield cushions equity holders from systematic risk
\end{itemize}

\vspace{0.5em}
\textbf{2. Lower expected returns}:
\begin{itemize}
\item Since $r_E = r_0 + (r_0 - r_D) \times \frac{D}{E} \times (1 - \tau_c)$, the leverage premium is smaller
\item Less risk $\rightarrow$ lower required return (CAPM)
\end{itemize}

\vspace{0.5em}
\textbf{3. Lower cash flows}:
\begin{itemize}
\item After-tax profits are $(1 - \tau_c) \times \text{EBIT}$
\item Higher taxes mean less cash available for dividends
\end{itemize}
\end{frame}

% Slide: Corporate taxes amplify equity market risk
\begin{frame}{Corporate tax rates amplify equity market risk}
\textbf{A macro perspective}: Since $\beta_E$ contains $(1 - \tau_c)$, \textcolor{uoftteal}{\textbf{higher corporate tax rates are a stabilizer for equity (stock) markets}}.

\vspace{1em}
\textbf{When $\tau_c$ falls} (e.g., tax cuts):
\begin{itemize}
\item $(1 - \tau_c) \uparrow$ $\rightarrow$ equity beta rises for levered firms
\item Stock market becomes \textbf{more sensitive} to macroeconomic shocks
\item Equity returns become \textbf{more volatile} during recessions and expansions
\end{itemize}

\vspace{1em}
\textbf{Broader spillover effects}:
\begin{enumerate}
\item \textbf{Wealth inequality}: Equity ownership concentrated among wealthy; higher volatility amplifies wealth swings
\item \textbf{Retirement stability}: Pensions hold equities; more systematic risk means less predictable retirement outcomes
\item \textbf{Financing opportunities}: Higher equity volatility raises cost of capital, making it harder to fund risky projects
\end{enumerate}
\end{frame}

% Slide: Policy comparison - US vs Canada
\begin{frame}{Policy comparison: US vs.\ Canada}
\textbf{Corporate tax environment}:
\begin{itemize}
\item \textbf{United States}: Lower corporate tax rates 
\item \textbf{Canada}: Higher corporate tax rates 
\end{itemize}

\vspace{1em}
\textbf{Implications for equity markets}:
\begin{itemize}
\item \textbf{US}: Lower $\tau_c$ $\rightarrow$ higher $(1 - \tau_c)$ $\rightarrow$ \textcolor{red}{\textbf{more systematic risk}} in equity markets
\item \textbf{Canada}: Higher $\tau_c$ $\rightarrow$ lower $(1 - \tau_c)$ $\rightarrow$ \textcolor{uoftteal}{\textbf{less systematic risk}} in equity markets
\end{itemize}

\vspace{1em}
\textbf{Historical performance}:
\begin{itemize}
\item US equity markets: Higher returns, higher volatility (more risk-taking, innovation)
\item Canadian equity markets: Lower returns, lower volatility (more stability)
\end{itemize}

\vspace{1em}
\textbf{The trade-off}:
\begin{itemize}
\item \textcolor{uoftteal}{\textbf{Innovation}}: Lower taxes $\rightarrow$ more risk-taking $\rightarrow$ growth and innovation
\item \textcolor{uoftblue}{\textbf{Stability}}: Higher taxes $\rightarrow$ less volatility $\rightarrow$ retirement security, less inequality
\end{itemize}
\end{frame}

%===============================================================================
% PART 3: CENTRAL BANKS AND CAPITAL STRUCTURE
%===============================================================================

\begin{frame}
\begin{center}
{\Huge \textcolor{uoftblue}{Part 3}}\\
\vspace{0.5em}
{\Large Central Banks, Interest Rates,\\and the Real Economy}
\end{center}
\end{frame}

% Slide: Why central banks matter for firms
\begin{frame}{Why do central banks matter for firms?}
We now have enough understanding in place to understand why central banks adjusting interest rates is so effective for adjusting the economic growth rate.

\vspace{1em}
\textbf{The connection to what we've learned}:
\begin{itemize}
\item Central banks control the \textcolor{uoftteal}{\textbf{policy rate}} (Bank of Canada's overnight rate, Fed's federal funds rate)
\item This affects $r_f$ (risk-free rate) and $r_D$ (cost of debt)
\item Which affects \textcolor{uoftblue}{\textbf{WACC}}
\item Which affects firm valuations and the \textcolor{red}{\textbf{NPV of investment projects}}
\item Which affects hiring, growth, and the real economy
\end{itemize}

\vspace{1em}
Central bank policy directly impacts corporate finance decisions we've been studying.
\end{frame}

% Slide: What is the policy rate?
\begin{frame}{What is the policy rate?}
\textbf{The policy rate} is the interest rate that the central bank sets and controls.

\vspace{1em}
\textbf{Different names in different countries}:
\begin{itemize}
\item \textbf{Canada}: Overnight rate (set by the Bank of Canada)
\item \textbf{United States}: Federal funds rate (set by the Federal Reserve)
\end{itemize}

\vspace{1em}
\textbf{What does it actually control?}
\begin{itemize}
\item The target rate for \textcolor{uoftteal}{\textbf{interbank lending}}
\item Banks lend reserves to each other overnight to meet regulatory requirements
\item The central bank sets the target for this rate
\end{itemize}

\vspace{1em}
\textbf{How can the central bank control this rate?}
\begin{itemize}
\item Conducts open market operations (buying/selling government bonds)
\item Adjusts the supply of reserves in the banking system
\item This pushes the interbank rate to the target level
\end{itemize}
\end{frame}

% Slide: Step 1 - Central bank sets policy rate
\begin{frame}{Step 1: Central bank sets the policy rate}
\textbf{The starting point}: Central bank announces a change to the policy rate: 1pp increase to combat inflation.

\vspace{1em}
\textbf{Example}: Bank of Canada raises overnight rate from 4\% to 5\%.

\vspace{1em}
\textbf{What is the overnight rate?}
\begin{itemize}
\item The target rate for banks lending to each other overnight
\item Banks need to borrow/lend reserves to meet daily requirements
\item This becomes the \textcolor{uoftteal}{\textbf{baseline cost of funds}} for banks
\end{itemize}

\vspace{1em}
\textbf{Key point}: When the overnight rate rises to 5\%, banks' opportunity cost of borrowing and lending funds with each other goes up to 5\%.
\end{frame}

% Slide: Step 2 - Banks pass on higher costs
\begin{frame}{Step 2: Banks pass on higher costs to borrowers}
\textbf{Question}: If banks' opportunity cost of funds goes up, what about the rate they charge private borrowers?

\vspace{1em}
\textbf{Answer}: It goes \textcolor{red}{\textbf{up}}.

\vspace{1em}
\textbf{Why?}
\begin{itemize}
\item Banks can now earn 5\% by lending to other banks (risk-free)
\item To lend to private borrowers (households, firms), they need to earn \emph{at least} 5\% + a risk premium
\item Otherwise, they'd just lend to other banks instead
\end{itemize}

\vspace{1em}
\textbf{Result}:
\begin{itemize}
\item \textbf{Mortgage rates} rise (households)
\item \textbf{Business loan rates} rise (firms)
\item \textbf{Prime rate} rises (banks' benchmark lending rate)
\end{itemize}

\vspace{1em}
Banks' higher opportunity cost $\rightarrow$ higher rates for all borrowers.
\end{frame}

% Slide: Step 3 - Private borrowers face higher opportunity cost
\begin{frame}{Step 3: Private borrowers' opportunity cost goes up}
\textbf{How does this affect the opportunity cost of private borrowers?}

\vspace{1em}
\textbf{Answer}: It goes \textcolor{red}{\textbf{up}}.

\vspace{1em}
\textbf{For households}:
\begin{itemize}
\item Can earn 5\% by holding government bonds (now yielding more)
\item Mortgage rates are higher
\item Less incentive to borrow, more incentive to save
\end{itemize}

\vspace{1em}
\textbf{For firms}:
\begin{itemize}
\item \textbf{Risk-free rate} ($r_f$): Government bond yields rise (baseline opportunity cost)
\item \textbf{Cost of debt} ($r_D$): Corporate bond yields and bank loans cost more
\item \textbf{Cost of equity} ($r_E$): Also rises (through CAPM, since $r_f \uparrow$)
\end{itemize}

\vspace{1em}
Everyone's opportunity cost is up $\rightarrow$ firms face higher cost of capital.
\end{frame}

% Slide: Step 4 - Higher cost of capital for firms
\begin{frame}{Step 4: Firms face higher cost of capital}
\textbf{Putting it together}:

\vspace{1em}
Central bank raises policy rate from 4\% to 5\%:

\vspace{0.5em}
\begin{enumerate}
\item Banks' opportunity cost $\uparrow$ (interbank rate = 5\%)
\item Banks charge private borrowers more
\item $r_f \uparrow$ (government bonds yield more)
\item $r_D \uparrow$ (corporate debt costs more)
\item $r_E \uparrow$ (through CAPM: $r_E = r_f + \beta_E \times (\mathbb{E}[R_m] - r_f)$)
\item \textcolor{red}{\textbf{WACC $\uparrow$}} (weighted average of $r_E$ and $r_D$ both rise)
\end{enumerate}

\vspace{1em}
\textbf{The chain}:
\[
\text{Policy rate} \uparrow \rightarrow r_f \uparrow, r_D \uparrow \rightarrow \text{WACC} \uparrow
\]

\vspace{1em}
Higher WACC means firms face a higher hurdle rate for investment projects.
\end{frame}

% Slide: Impact on investment and NPV
\begin{frame}{How policy rates affect investment decisions}
\textbf{Recall}: A project is worth pursuing if NPV $> 0$.

\vspace{1em}
NPV depends on the discount rate (WACC):
\[
\text{NPV} = \sum_{t=1}^{T} \frac{CF_t}{\text{WACC}^t} - \text{Initial Investment}
\]

\vspace{1em}
\textbf{When the central bank raises rates}:
\begin{itemize}
\item WACC $\uparrow$ $\rightarrow$ discount rate increases
\item NPV of projects $\downarrow$ (future cash flows worth less today)
\item \textcolor{red}{\textbf{Fewer projects have positive NPV}}
\item Investment space shrinks $\rightarrow$ firms invest less
\item Lower investment, fewer projects $\rightarrow$ economic growth \textbf{slows} (inflation reduces)
\end{itemize}
\end{frame}

% Slide: Impact on WACC and firm value
\begin{frame}{How policy rates affect firm valuations}
\textbf{Recall}: $\text{WACC} = r_0 \times \left[1 - \frac{D}{E+D} \times \tau_c\right]$

\vspace{1em}
And $r_0$ depends on $r_f$ (through CAPM): $r_0 = r_f + \beta_A \times (\mathbb{E}[R_m] - r_f)$

\vspace{1em}
\textbf{When the central bank raises the policy rate}:
\begin{enumerate}
\item Policy rate $\uparrow$ $\rightarrow$ $r_f \uparrow$ and $r_D \uparrow$
\item $r_f \uparrow$ $\rightarrow$ $r_0 \uparrow$ (unlevered cost of capital rises)
\item WACC $\uparrow$ (discount rate increases)
\item Firm value $= \sum_t CF_t / \text{WACC}^t$ $\downarrow$ (present value of cash flows falls)
\end{enumerate}

\vspace{1em}
\textbf{When the central bank cuts the policy rate}:
\begin{itemize}
\item Everything reverses: WACC $\downarrow$, firm value $\uparrow$
\end{itemize}

\vspace{1em}
This is why stock markets react immediately to central bank announcements.
\end{frame}

% Slide: Real economy effects
\begin{frame}{Real economy effects: Investment, hiring, growth}
\textbf{The full transmission to the real economy}:

\vspace{1em}
\textbf{Central bank raises rates}:
\begin{enumerate}
\item Policy rate $\uparrow$ $\rightarrow$ $r_f$, $r_D$, WACC all rise
\item Fewer projects have positive NPV $\rightarrow$ \textcolor{red}{\textbf{less investment}}
\item Less capital spending (factories, equipment, R\&D)
\item Firms hire fewer workers $\rightarrow$ \textcolor{red}{\textbf{unemployment rises}}
\item Economic growth slows $\rightarrow$ inflation cools
\end{enumerate}

\vspace{1em}
\textbf{Central bank cuts rates}:
\begin{enumerate}
\item Policy rate $\downarrow$ $\rightarrow$ $r_f$, $r_D$, WACC all fall
\item More projects have positive NPV $\rightarrow$ \textcolor{uoftteal}{\textbf{more investment}}
\item More capital spending
\item Firms hire more workers $\rightarrow$ \textcolor{uoftteal}{\textbf{unemployment falls}}
\item Economic growth accelerates $\rightarrow$ may fuel inflation
\end{enumerate}

\vspace{1em}
This is the \textcolor{uoftblue}{\textbf{real channel}} of monetary policy: interest rates $\rightarrow$ investment $\rightarrow$ employment $\rightarrow$ growth.
\end{frame}

%===============================================================================
% PART 4: HEDGING AND RISK MANAGEMENT
%===============================================================================

\begin{frame}
\begin{center}
{\Huge \textcolor{uoftblue}{Part 4}}\\
\vspace{0.5em}
{\Large Introduction to Hedging}
\end{center}
\end{frame}

% Slide: Hedging vs speculation
\begin{frame}{Hedging vs.\ speculation}
\textbf{Two ways to use financial instruments}:

\vspace{1em}
\textbf{1. Hedging}:
\begin{itemize}
\item Using financial instruments to \textcolor{uoftteal}{\textbf{reduce a specific risk exposure}}
\item Goal: Protect against adverse price movements
\item Example: An airline worried about rising oil prices
\end{itemize}

\vspace{1em}
\textbf{2. Speculation}:
\begin{itemize}
\item Using financial instruments to \textcolor{red}{\textbf{increase a specific risk exposure}}
\item Goal: Profit from anticipated price movements
\item Example: A trader betting oil prices will rise
\end{itemize}

\vspace{1em}
\textbf{Key difference}: Hedgers want to \emph{eliminate} a risk, speculators want to \emph{take on} a risk.

\vspace{1em}
Today we focus on \textbf{hedging}---how firms manage their risk exposures.
\end{frame}

% Slide: Motivating example - airline
\begin{frame}{Motivating example: An airline's oil price exposure}
\textbf{Consider an airline}:

\vspace{1em}
\textbf{The problem}:
\begin{itemize}
\item Fuel is a major operating cost (20-30\% of expenses)
\item Oil prices are volatile
\item \textcolor{red}{\textbf{Rising oil prices hurt profitability}}
\begin{itemize}
    \item Higher fuel costs $\rightarrow$ lower profit margins
    \item Can't always pass costs to customers immediately
\end{itemize}
\end{itemize}

\vspace{1em}
\textbf{The exposure}:
\begin{itemize}
\item When oil prices $\uparrow$, airline's costs $\uparrow$, profits $\downarrow$
\item When oil prices $\downarrow$, airline's costs $\downarrow$, profits $\uparrow$
\end{itemize}

\vspace{1em}
\textbf{Question}: How can the airline reduce this exposure to oil price shocks?
\end{frame}

% Slide: Naive solution - buy oil stocks
\begin{frame}{Naive solution: Buy oil company stocks?}
\textbf{Idea}: Buy shares of oil companies (e.g., Exxon, Shell).

\vspace{1em}
\textbf{Logic}:
\begin{itemize}
\item When oil prices $\uparrow$, oil company stocks $\uparrow$
\item Gains on oil stocks offset losses from higher fuel costs
\end{itemize}

\vspace{1em}
\textbf{Why this doesn't work well}:
\begin{enumerate}
\item \textbf{Beta exposure}: Oil stocks have systematic risk
\begin{itemize}
    \item Stock prices move with the market, not just oil prices
    \item In a recession, oil stocks can fall even if oil prices rise
\end{itemize}

\vspace{0.5em}
\item \textbf{Downside risk}: You can lose money on the stocks
\begin{itemize}
    \item If oil prices fall, you lose on both the stocks \emph{and} your core business doesn't benefit as much as stocks lose
\end{itemize}

\vspace{0.5em}
\item \textbf{Imperfect correlation}: Oil stock returns $\neq$ oil price changes
\end{enumerate}

\vspace{1em}
\textbf{What we want}: An instrument that pays off \emph{only} when oil prices rise, and nothing otherwise.
\end{frame}

% Slide: Introduce options
\begin{frame}{A better solution: Oil call options}
\textbf{What if we could design an instrument that}:
\begin{itemize}
\item Pays us when oil prices rise (offsetting higher fuel costs)
\item Has limited downside when oil prices fall
\end{itemize}

\vspace{1em}
This is a \textcolor{uoftteal}{\textbf{call option}} on oil.

\vspace{1em}
\textbf{Formal definition of a call option}:
\begin{itemize}
\item Gives the holder the \textbf{right} (not obligation) to \textbf{buy} an asset at a specified price $K$ (called the \textbf{strike price})
\item \textbf{Expiration date} ($T$): When the option expires
\item \textbf{Premium} ($C$): The upfront price paid to purchase the option
\end{itemize}

\vspace{1em}
\textbf{Exercise decision at expiration}:
\begin{itemize}
\item If oil price $S_T > K$: Exercise the option (buy at $K$, savings $= S_T - K$) 
\item If oil price $S_T < K$: Let the option expire (walk away, savings = 0)
\end{itemize}
\end{frame}

% Slide: Payoff vs profit
\begin{frame}{Payoff vs.\ profit}
\textbf{Important distinction}: Payoff $\neq$ Profit.

\vspace{1em}
\textbf{Payoff at expiration}:
\[
\text{Payoff} = \max(S_T - K, 0)
\]
\begin{itemize}
\item This is what you receive at expiration
\item Ignores the upfront premium paid
\end{itemize}

\vspace{1em}
\textbf{Profit at expiration}:
\[
\text{Profit} = \max(S_T - K, 0) - C
\]
\begin{itemize}
\item This accounts for the premium $C$ paid upfront
\item Can be negative (you lose the premium if option expires worthless)
\end{itemize}

\vspace{1em}
\textbf{Key feature}: Payoff is \textcolor{uoftteal}{\textbf{asymmetric}}---you benefit from upside, but maximum loss is limited to the premium $C$.
\end{frame}

% Slide: Call option payoff diagram
\begin{frame}{Call option payoff: The hockey stick}
\begin{center}
\begin{tikzpicture}[scale=0.8]
% Axes
\draw[->] (0,0) -- (8,0) node[right] {Oil price at expiration};
\draw[->] (0,-1) -- (0,4) node[above] {Payoff};

% Strike price line
\draw[dashed] (4,-1) -- (4,4);
\node[below] at (4,0) {$K$};

% Payoff line
\draw[very thick, uoftteal] (0,0) -- (4,0);
\draw[very thick, uoftteal] (4,0) -- (7,3);

% Labels
\node[above right] at (6,2.5) {\textcolor{uoftteal}{Payoff $= \max(S - K, 0)$}};
\node[below] at (2,-0.5) {Option expires worthless};
\node[above] at (6,0.5) {Exercise option, profit};

% Mark zero
\draw (0,0.1) -- (0,-0.1) node[left] {0};
\end{tikzpicture}
\end{center}

\vspace{0.5em}
\textbf{The ``hockey stick'' shape}:
\begin{itemize}
\item If $S < K$: Payoff $= 0$ (don't exercise)
\item If $S > K$: Payoff $= S - K$ (exercise and profit)
\end{itemize}

\vspace{0.5em}
\textbf{For the airline}: This caps downside exposure to oil price increases.
\end{frame}

% Slide: Airline uses call options
\begin{frame}{How the airline uses call options to hedge}
\textbf{Hedging strategy}:

\vspace{1em}
\textbf{Airline's position}:
\begin{enumerate}
\item \textbf{Underlying exposure}: Rising oil prices hurt profits
\item \textbf{Hedge}: Buy call options on oil for premium $C$ with strike $K$
\end{enumerate}

\vspace{1em}
\textbf{Outcome}:
\begin{itemize}
\item \textbf{If oil price rises above $K$}:
\begin{itemize}
    \item Fuel costs increase, hurting profits
    \item Call option pays off $(S - K)$, offsetting the loss
    \item Net effect: Capped exposure to oil price increases
\end{itemize}

\vspace{0.5em}
\item \textbf{If oil price stays below $K$}:
\begin{itemize}
    \item Fuel costs remain manageable
    \item Call option expires worthless (sunk cost of premium)
    \item Net effect: Small loss (premium paid), but no major impact
\end{itemize}
\end{itemize}

\vspace{1em}
\textbf{Key insight}: The airline has \textcolor{uoftteal}{\textbf{insurance}} against rising oil prices.
\end{frame}

% Slide: Preview next week
\begin{frame}{Next week: Options pricing}
\textbf{Today}: Introduction to hedging and options (hockey sticks, protective puts, covered calls)

\vspace{1em}
\textbf{Next week}: How do we \emph{price} these options?

\vspace{1em}
\textbf{The fundamental challenge}:
\begin{itemize}
\item Pricing is about computing present value: $C = \text{PV}(\text{Expected Payoff})$
\item But there's \textcolor{red}{\textbf{so much uncertainty}} in the payoff!
\item Payoff depends on future stock price: $\max(S_T - K, 0)$
\item Estimating the expected PV is extremely difficult
\end{itemize}

\vspace{1em}
\textbf{The solution}: We need new tricks:
\begin{itemize}
\item \textcolor{uoftteal}{\textbf{No-arbitrage arguments}}: If two portfolios have the same payoff, they must have the same price
\item \textcolor{uoftblue}{\textbf{Replicating (synthetic) portfolios}}: Build an option payoff using stocks and bonds
\end{itemize}

\vspace{1em}
\textbf{Next week's topics}: Put-call parity, binomial trees, Black-Scholes formula.
\end{frame}

%===============================================================================
% CONCLUSION
%===============================================================================

\begin{frame}{Conclusion: Connecting the Pieces}
\begin{enumerate}
\item \textbf{MM Theorem}: Capital structure doesn't matter in a frictionless world (the benchmark)
\item \textbf{Taxes \& Bankruptcy}: Real-world frictions create optimal leverage
\item \textbf{Central Banks}: Policy rates affect WACC, investment, and the real economy
\item \textbf{Hedging}: Options let firms manage specific risk exposures
\end{enumerate}

\vspace{1em}
\textbf{The big picture}:
\begin{itemize}
\item Firm-level decisions (capital structure, hedging) are deeply connected to macro conditions (tax policy, monetary policy)
\item Systematic risk is the common thread: it affects optimal leverage, equity market volatility, and central bank effectiveness
\item Corporate finance isn't just about individual firms---it's about how the entire economy allocates risk and capital
\end{itemize}

\vspace{1em}
\textbf{Next week}: Options pricing (put-call parity, binomial trees, Black-Scholes).
\end{frame}

\begin{frame}
\begin{center}
{\Huge Thank you!}

\vspace{2em}
{\Large Questions?}
\end{center}
\end{frame}


\end{document}
