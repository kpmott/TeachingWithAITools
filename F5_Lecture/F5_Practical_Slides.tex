% %!TEX TS-program = pdflatex
\documentclass[aspectratio=169,11pt]{beamer}
\usepackage{../eng_beamer}

% Title page information
\title[JRE 300 Finance Lecture 5 Practical]{Finance Lecture 5 Practical}
\subtitle{JRE 300: Fundamentals of Accounting and Finance}
\author{Kevin Mott}
\institute{Rotman School of Management\\University of Toronto}
\date{}

\begin{document}

% Title slide
\begin{frame}
\titlepage
\end{frame}

% Quick Overview
\begin{frame}{Quick Overview}
\textbf{The big ideas from Lecture 5:}
\begin{enumerate}
\item MM Theorem: Capital structure doesn't matter (in a frictionless world)
\item How leverage affects equity beta and cost of equity
\item Taxes create a debt advantage (interest tax shield)
\item Bankruptcy costs create a debt disadvantage
\item Trade-off theory: Balancing tax benefits vs.\ distress costs
\item Central banks affect WACC and investment through policy rates
\end{enumerate}

\vspace{1em}
\textbf{Today}: Quick review + key formulas + practice problems
\end{frame}

% Part 1: MM and Capital Structure
\begin{frame}
\begin{center}
{\Huge \textcolor{uoftblue}{MM Theorem}}\\[1em]
{\Large Capital Structure Basics}
\end{center}
\end{frame}

\begin{frame}{Firm value: $V = E + D$}
\textbf{A firm's total value}:
\[
\boxed{V = E + D}
\]

where:
\begin{itemize}
\item $V$ = firm value
\item $E$ = \textbf{market value} of equity
\item $D$ = \textbf{market value} of debt
\end{itemize}

\vspace{1em}
\textbf{Key insight}: Firm value comes from assets (NPV $>$ 0 projects), not from how they're financed.

\vspace{1em}
\textbf{MM Theorem} (frictionless world):
\begin{itemize}
\item Capital structure doesn't affect firm value
\item WACC stays constant regardless of $D/E$ ratio
\item Changing leverage just reallocates risk among investors
\end{itemize}
\end{frame}

\begin{frame}{How leverage affects equity beta}
\textbf{Beta conservation equation}:
\[
\boxed{\beta_E = \beta_{\text{Project}} + \frac{D}{E}(\beta_{\text{Project}} - \beta_D)}
\]

\vspace{1em}
\textbf{With risk-free debt} ($\beta_D = 0$):
\[
\beta_E = \beta_{\text{Project}} \times \left(1 + \frac{D}{E}\right)
\]

\vspace{1em}
\textbf{Example}: $\beta_{\text{Project}} = 1.2$, $D/E = 1$
\[
\beta_E = 1.2 \times (1 + 1) = 2.4
\]

\vspace{0.5em}
\textbf{Intuition}: More debt $\rightarrow$ equity becomes riskier $\rightarrow$ $\beta_E$ increases
\end{frame}

\begin{frame}{Cost of equity rises with leverage}
\textbf{Without taxes}:
\[
\boxed{r_E = r_0 + (r_0 - r_D) \times \frac{D}{E}}
\]

where $r_0$ = unlevered cost of equity

\vspace{.5em}
\textbf{Example}: $r_0 = 11.4\%$, $r_D = 3\%$, $D/E = 1$
\[
r_E = 11.4\% + (11.4\% - 3\%) \times 1 = 19.8\%
\]

\vspace{.5em}
\textbf{With taxes} ($\tau_c = $ corporate tax rate):
\[
\boxed{r_E = r_0 + (r_0 - r_D) \times \frac{D}{E} \times (1 - \tau_c)}
\]

\vspace{0.5em}
Tax shield reduces equity risk, so leverage premium is smaller.
\end{frame}

\begin{frame}{WACC without taxes}
\textbf{Formula}:
\[
\text{WACC} = \frac{E}{E+D} \times r_E + \frac{D}{E+D} \times r_D
\]

\vspace{1em}
\textbf{MM Result}: In a frictionless world,
\[
\boxed{\text{WACC} = r_0}
\]
regardless of capital structure!

\vspace{1em}
\textbf{Why?} Rising $r_E$ exactly offsets the use of cheaper debt.

\end{frame}

% Part 2: Taxes
\begin{frame}
\begin{center}
{\Huge \textcolor{uoftblue}{Taxes}}\\[1em]
{\Large The Interest Tax Shield}
\end{center}
\end{frame}

\begin{frame}{Interest tax shield example}
\textbf{Two identical firms}, EBIT = \$100, $\tau_c = 25\%$
\begin{itemize}
\item \textbf{Firm A}: No debt
\item \textbf{Firm B}: \$500 debt at 5\% interest
\end{itemize}

\vspace{1em}
\begin{center}
\begin{tabular}{lcc}
\toprule
& \textbf{Firm A} & \textbf{Firm B} \\
\midrule
EBIT & \$100 & \$100 \\
Interest & \$0 & \$25 \\
Taxable income & \$100 & \$75 \\
Taxes (25\%) & \$25 & \$18.75 \\
\midrule
\textbf{Cash to investors} & \textbf{\$75} & \textbf{\$81.25} \\
\bottomrule
\end{tabular}
\end{center}

\vspace{1em}
\textbf{Tax shield} $= $ Interest $\times \tau_c = \$25 \times 0.25 = \$6.25$

\vspace{0.5em}
Debt saves taxes! This is why capital structure matters in the real world.
\end{frame}

\begin{frame}{WACC with taxes}
\textbf{Formula with tax shield}:
\[
\boxed{\text{WACC} = \frac{E}{E+D} \times r_E + \frac{D}{E+D} \times r_D \times (1 - \tau_c)}
\]

\vspace{1em}
\textbf{Key result}: More debt $\rightarrow$ lower WACC
\[
\text{WACC} = r_0 \times \left[1 - \frac{D}{E+D} \times \tau_c\right]
\]

\vspace{1em}
\textbf{Why lower WACC matters}:
\begin{itemize}
\item Lower discount rate $\rightarrow$ higher firm value
\item More projects have positive NPV (lower hurdle rate)
\end{itemize}

\end{frame}

% Part 3: Bankruptcy
\begin{frame}
\begin{center}
{\Huge \textcolor{uoftblue}{Bankruptcy Costs}}\\[1em]
{\Large Why Not 100\% Debt?}
\end{center}
\end{frame}

\begin{frame}{Types of bankruptcy costs}
\textbf{1. Direct costs}:
\begin{itemize}
\item Legal fees, court costs, administrative expenses
\item Typically 3-7\% of firm value
\end{itemize}

\vspace{1em}
\textbf{2. Indirect costs} (often much larger!):
\begin{itemize}
\item \textbf{Lost customers}: Worry about warranties, service
\item \textbf{Lost suppliers}: Demand cash, refuse credit
\item \textbf{Lost employees}: Talented workers leave
\item \textbf{Underinvestment}: Can't raise capital for good projects
\item \textbf{Asset fire sales}: Forced to sell below fair value
\end{itemize}

\vspace{1em}
\textbf{Key point}: More debt $\rightarrow$ higher bankruptcy risk $\rightarrow$ higher expected bankruptcy costs
\end{frame}

\begin{frame}{Trade-off theory of capital structure}
\textbf{The optimization}:
\[
\max_{D/E} \quad \text{PV(Tax Shield)} - \text{PV(Bankruptcy Costs)}
\]

\vspace{1em}
\textbf{Trade-off}:
\begin{itemize}
\item \textcolor{uoftteal}{\textbf{More debt}}: Lower WACC, higher tax shield
\item \textcolor{red}{\textbf{More debt}}: Higher bankruptcy risk, destroys value
\end{itemize}

\vspace{1em}
\textbf{Optimal capital structure}: Balance tax benefits against distress costs

\vspace{1em}
\textbf{Depends on}:
\begin{itemize}
\item Firm's systematic risk ($\beta$)
\item Cash flow stability
\item Asset tangibility
\item Tax rates
\end{itemize}
\end{frame}

\begin{frame}{Corporate taxes and equity market risk}
\textbf{Recall}: $\beta_E = \beta_{\text{Project}} + \frac{D}{E}(\beta_{\text{Project}} - \beta_D) \times (1 - \tau_c)$

\vspace{1em}
\textbf{Macro implication}: Higher corporate tax rates stabilize equity markets!

\vspace{1em}
\textbf{When $\tau_c$ falls} (tax cuts):
\begin{itemize}
\item $(1 - \tau_c) \uparrow$ $\rightarrow$ equity beta rises for levered firms
\item Stock market becomes more volatile
\item Equity returns more sensitive to macro shocks
\end{itemize}

\vspace{1em}
\textbf{Spillover effects}:
\begin{itemize}
\item \textbf{Wealth inequality}: More volatile equity markets amplify wealth swings
\item \textbf{Retirement stability}: More systematic risk in pensions
\item \textbf{Financing}: Higher volatility raises cost of capital
\end{itemize}
\end{frame}

% Part 4: Central Banks
\begin{frame}
\begin{center}
{\Huge \textcolor{uoftblue}{Central Banks}}\\[1em]
{\Large Monetary Policy and WACC}
\end{center}
\end{frame}

\begin{frame}{The transmission mechanism}
\textbf{How central banks affect the real economy}:

\vspace{1em}
\begin{enumerate}
\item \textbf{Central bank} sets policy rate (overnight rate, fed funds rate)
\item \textbf{Banks} pass on higher costs to borrowers
\item \textbf{$r_f$ and $r_D$} both rise (government and corporate bonds)
\item \textbf{$r_E$} rises (through CAPM: $r_E = r_f + \beta_E \times$ market premium)
\item \textbf{WACC} rises (weighted average of $r_E$ and $r_D$)
\item \textbf{NPV} of projects falls (higher discount rate)
\item \textbf{Investment} decreases (fewer positive NPV projects)
\item \textbf{Hiring} decreases, unemployment rises
\item \textbf{Economic growth} slows, inflation cools
\end{enumerate}

\vspace{1em}
\textbf{The chain}:
\[
\text{Policy rate} \uparrow \rightarrow \text{WACC} \uparrow \rightarrow \text{NPV} \downarrow \rightarrow \text{Investment} \downarrow
\]
\end{frame}

\begin{frame}{Policy rates affect firm valuations}
\textbf{Firm value}:
\[
V = \sum_{t=1}^{\infty} \frac{CF_t}{\text{WACC}^t}
\]

\vspace{.5em}
\textbf{When central bank raises rates}:
\begin{itemize}
\item WACC $\uparrow$ (higher discount rate)
\item Firm value $\downarrow$ (PV of cash flows falls)
\item Stock prices fall immediately
\end{itemize}

\vspace{.5em}
\textbf{When central bank cuts rates}:
\begin{itemize}
\item WACC $\downarrow$ (lower discount rate)
\item Firm value $\uparrow$ (PV of cash flows rises)
\item Stock prices rise immediately
\end{itemize}

\vspace{.5em}
\textbf{This is why}: Markets react instantly to central bank announcements!
\end{frame}

% Key formulas
\begin{frame}
\begin{center}
{\Huge \textcolor{uoftblue}{Key Formulas}}
\end{center}
\end{frame}

\begin{frame}{Capital structure formulas}
\textbf{Firm value}:
\[
V = E + D
\]

\textbf{Equity beta with leverage}:
\[
\beta_E = \beta_{\text{Project}} + \frac{D}{E}(\beta_{\text{Project}} - \beta_D)
\]

\textbf{Cost of equity with leverage (with taxes)}:
\[
r_E = r_0 + (r_0 - r_D) \times \frac{D}{E} \times (1 - \tau_c)
\]

\textbf{WACC with taxes}:
\[
\text{WACC} = \frac{E}{E+D} \times r_E + \frac{D}{E+D} \times r_D \times (1 - \tau_c)
\]
\end{frame}

\begin{frame}
\begin{center}
{\Huge \textcolor{uoftblue}{Practice Problems}}
\end{center}
\end{frame}

\end{document}
