\item \textbf{\color{rotman} Project Financing and NPV} \\ 
A company is evaluating the construction of a new battery plant. The project requires an upfront investment of \$300 million and will produce perpetual annual EBIT of \$40 million. The corporate tax rate is 30\% and the unlevered cost of capital is 10\%. 

\begin{enumerate}

\item Calculate the NPV of the project under an all-equity financed structure. 

\end{enumerate}

The company instead considers financing the project through a capital structure that is equal parts debt and equity. The interest rate on the debt is 5\% and the total debt outstanding would be \$150 million. 

\begin{enumerate}[resume]

\item Calculate the new cost of equity under this capital structure. 

\item Compute the WACC under the new capital structure. 

\item Recalculate the NPV of the project under this financing plan using the WACC you calculated in part (c). 

\item Is there a real effect of choosing different financing streams? In finance, we use the term \emph{real} to mean production-based outcomes. Hint: compare your answers to (a) and (d) and the implications for if the battery plant should be built. 

\end{enumerate}

\sol{

\begin{enumerate}

\item The unlevered cost of capital is 10\%, so using the perpetuity formula the NPV in millions is:
\[\text{NPV} = -300 + \frac{40\times(1-0.30)}{0.10} = -20\]

\item Under the new capital structure:
\[r_E = r_0 + \frac{D}{E} \times (1-\tau_c) \times (r_0 - r_D) = 0.10 + \frac{1}{1} \times (1-0.30)\times(0.10 - 0.05) = 13.5\%\]

\item The new WACC is 
\[\text{WACC} = \frac{E}{E+D} \times r_E + \frac{D}{E+D} \times r_D (1-\tau_c) = \frac{1}{2} \times 0.135 + \frac{1}{2} \times 0.05 \times (1-0.3) = 8.5\%\]

\item Using the WACC from part (c), the NPV in millions is:
\[\text{NPV} = -300 + \frac{40\times(1-0.30)}{0.085} = -300 + \frac{28}{0.085} = 29.41\]

\item Yes, there is a real effect of financing. Under all equity, the project has a negative NPV. But the tax shield of debt creates enough value to make the project overall positive NPV when financed with a capital structure of equal parts debt and equity. The availability of debt financing allows the company to create value, producing batteries after the plant is built. 

\end{enumerate}

}