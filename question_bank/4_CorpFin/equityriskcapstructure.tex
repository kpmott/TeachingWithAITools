\item \textbf{\color{rotman} Capital Structure and Equity Risk}

A company is 100\% equity-financed and its stock has a historical beta of 3.

\begin{enumerate}
\item The president of the country in which the company is headquartered threatens to cause a recession, and the stock market is down 5\% as a result. What is the expected change in the return of this company?

\item What happens to the equity beta if the company decides to lever up to a capital structure of 75\% equity, 25\% debt? What would the expected change be in the return of the company after the President's announcement?

\item What happens to the equity beta if the company decides to lever up to a capital structure of 25\% equity, 75\% debt? What would the expected change be in the return of the company after the President's announcement?

\item If you were a shareholder in the company, which capital structure do you prefer of (a)-(c)? Hint: don't just look at the downside risk analyzed in (a)-(c).

\end{enumerate}

\sol{
\begin{enumerate}
\item Using the beta of 3:
\[
\Delta R = \beta \times \Delta R_m = 3 \times (-5\%) = -15\%
\]
The expected return change is \textbf{-15\%}.

\item The new equity beta is given by:
\[
\beta_{new} = \beta_{old} \times \left(1 + \frac{D}{E}\right) = 3 \times \left(1 + \frac{0.25}{0.75}\right) = 4
\]
New return change:
\[
\Delta R = 4 \times (-5\%) = -20\%
\]

\item New equity beta:
\[
\beta_{new} = 3 \times \left(1 + \frac{0.75}{0.25}\right) = 3 \times 4 = 12
\]
New return change:
\[
\Delta R = 12 \times (-5\%) = -60\%
\]

\item As a shareholder, the choice of capital structure depends on your risk tolerance and expected return preferences. While the higher debt leads to a higher expected stock price decline in times of economic uncertainty, the higher debt means higher beta, which means higher expected returns (over long horizons).

\end{enumerate}
}